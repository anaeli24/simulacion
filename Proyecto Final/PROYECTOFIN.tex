\documentclass[10pt,twocolumn]{article}

\usepackage[spanish]{babel}
\usepackage[numbers,sort&compress]{natbib}
\usepackage[T1]{fontenc}
\usepackage[ansinew]{inputenc}
\usepackage{graphicx}
\usepackage{url}
\usepackage{subcaption}
\usepackage{caption}
\usepackage{listings}
\usepackage{enumerate}
\usepackage{amsmath}
\usepackage{float}
\usepackage[numbers,sort&compress]{natbib}

\begin{document}

\twocolumn[
  \begin{@twocolumnfalse}

\title{\textbf{Simulacion de Dinamica Molecular con potencial de Lenard Jones}}
\author{Anahi Elizabeth Llano}

  \maketitle
    \begin{abstract}
      \begin{center}
      Aqui va el resumen del trabajo describe de que se trata el trabajo, que en si se hace, como se evalua y que resultados se obtiene.
      \end{center}
    \end{abstract}
    Palabras clave: \textit{Insertar palabras clave aqui.\\ \\}
  \end{@twocolumnfalse}
  ]


\section{Introduccion}\label{intro}

 Se describe de que de trata el estudio y se explica su motivacion. Por lo general se plantea una hipotesis (o varias) y se explican los objetivos (contribuciones esperadas) del trabajo. No se utiliza aun mucha terminologia tecnica ni notacion formal y no se entra en detalle a nada que no es esencial. Al final de la introduccion viene un parrafo que describe la estructura del resto del reporte, haciendo referencia a las demas secciones.

\section{Antecedentes}\label{antes}
 La informacion referente al tema conceptos basicos en este caso como es la dinamica molecular , el potencial de lenard jones, para que quede mas entendible el tema. 

\section{Trabajos relacionados}\label{trabajos}
Algunos trabajos relacionados que se han desarrollado sobre el tema de interes

\section{Metodolog\'{i}a}\label{met}
Se partira del codigo de la practica 9 que es el de interaccion entre particulas modificando el potencial gravitacional descartando este y tomando en cuenta el potencial de lenard para llevar a cabo una simulacion de dinamica molecular en un liquido.

\section{Resultados y Discusi\'{o}n}\label{res}



\section{Conclusi\'{o}n}\label{con}


  \bibliography{PROYECTOFIN}
  \bibliographystyle{plainnat}
\end{document}